% Klassifiziert den Dokumenten-Typ
% Doku: http://exp1.fkp.physik.tu-darmstadt.de/tuddesign/
% Farben: http://www.tu-darmstadt.de/media/medien_stabsstelle_km/services/medien_cd/das_bild_der_tu_darmstadt.pdf
%  bigchapter: Chapter haben doppelte Schriftgröße
%  linedtoc: Linien im Inhaltsverzeichnis wie bei Überschriften
%  colorbacktitle: Der Dokumenten-Titel wird mir der Accentfarbe hinterlegt
\documentclass[bigchapter,colorback,accentcolor=tud4b,linedtoc,11pt]{tudreport}

% Input Dokument hat das Encoding UTF-8
\usepackage[utf8]{inputenc}
% Wichtiges Paket für Links und verlinktes Inhaltsverzeichnis
\usepackage[ngerman]{hyperref}
% Paket für Fußnoten
\usepackage[stable]{footmisc}
% Paket für amsmath (aligned mathe formeln)
\usepackage{amsmath}
% Paket für Bibliotheks-Verzeichnis, square: Verwende eckige statt runde klammern
% \usepackage[square]{natbib}
% Paket zum Plotten von Datensätzen
\usepackage{pgfplots}
\pgfkeys{
  /pgfplots/PolarStyle/.style={
    ylabel=Leistung in W,
    width=0.33\linewidth,
    height=0.33\linewidth,
    scale only axis,
    grid=both,
    tick align=outside,
    tickpos=left,
    minor x tick num=2,
    minor y tick num=1
  }
}

\usetikzlibrary{pgfplots.polar}
% Verwende deutsche Bezeichner für Inhaltsverzeichnis, ... (ngerman = New German: neue Rechtschreibung)
\usepackage{ngerman}
% Deutsche Zahlen (entfernt z.B. das Leerzeichen nach einem Dezimal-Komma)
\usepackage{ziffer} 

\usepackage[verbose]{placeins}

%wegen Grafikverschiebung hinzugefügt
\usepackage{float}

%\usepackage{graphicx}
%\usepackage{caption}
\usepackage{subcaption} %Für subfigures

% PDF-Optionen
\hypersetup{
  pdftitle={TU Darmstadt \- Physikalisches Praktikum für Fortgeschrittene},
  pdfauthor={Esra Bauer und Sören Link},
  pdfsubject={Versuch 3.3-B},
  pdfview=FitH,
}
% Nummeriere formeln in Subsections einzeln
% Kleines makro zur assymetrischen Fehlerangabe

% Entspricht-Zeichen
\usepackage{scalerel}

\newcommand\equalhat{%
\let\savearraystretch\arraystretch
\renewcommand\arraystretch{0.3}
\begin{array}{c}
\stretchto{
    \scalerel*[\widthof{=}]{\wedge}
    {\rule{1ex}{3ex}}%
}{0.5ex}\\ 
=%
\end{array}
\let\arraystretch\savearraystretch
}
%BEGINN TITELSEITE

\title{Polarisationsanalyse von Licht mittels Stokes-Formalismus}

\subtitle{Esra Bauer  \\Sören Link}

\subsubtitle{Betreuer: David Rupp \hfill Versuchsdatum: 10. November 2014}

\author{Esra Bauer, Sören Link}

%\settitlepicture{img/title.jpg}

\institution{Physikalisches Praktikum \\für Fortgeschrittene \\ Versuch 3.3-B}

\date{\today}


%ENDE TITELSEITE

\begin{document}
%ANFANG DOKUMENT

%Titelseite einfügen
\maketitle

%Inhaltsverzeichnis einfügen
\tableofcontents

%ANFANG INHALT

\chapter{Einleitung}
TODO: Einleitung zur Röntgenkleinwinkelstreuung

\chapter{Grundlagen}
\section{Polymere}
TODO: Grundlagen Polymere (langkettig etc.)

\section{Glasübergang}
TODO: Glasübergang vs. Kristallisation

\section{Kristallinität in Polymeren}
TODO: Warum kristallisieren polymere? Keimbildung, Teilkristallinität, $F = U - TS$ (Van der Vaals-Kräfte), Sphärolite, warum keine vollständige Kristallisation?

\section{Rötgenstreuung}
TODO: Funktionsweie Röntgenröhren, Peaks, Streuung (Amplitude, Intensität, Furier, Autokorrelationsunfktion, Elektronendichte)


\section{Streuung an zweiphasigen Schichtsystemen}
TODO: Autokorrelationsfunktion, Invariante Q, Herleitung (wie in vorbesprechung), Abweichung Tatsächliche Korrelationsfunktion von einer für ein idealisiertes System


\chapter{Durchführung}
\section{Präparation der PET-Probe}
TODO: Vorgehen, eingestellte Temperatur (idealerweie knapp unter 300°C), gemachte Fehler (zu viel PET "Rohmaterial"`, zu hohe Temperatur -> verfärbung)

\section{Bestimmung des Primärstrahlprofils}
TODO: Vorgehen, einstellungen, Warum? (Entschlierung)

\section{Hintergrundmessung mit leerem Probenhalter und Aluminiumfolie}
TODO: Vorgehen (moving slit), Aufbaue (leere Probenhalter, Alufolie)

\section{Messungen mit Probe}
TODO: Messreihenfolge (Primärstrahlmessung(PS) -> Wanderspalt (WS) -> PS -> WS -> \dots; 170grad -> \dots)

\chapter{Auswertung}
\section{Strahlcharakterisierung}
TODO: Darstellung des Primärstrahls

\section{PET-Streudaten}
TODO: Rohdaten, Entschmierung, Intensität vs. Streuvektor, Langperiode abschätzen

\section{Eindimensionale Korrelationsfunkton}
TODO: Berechnung/Darstellung von K(Z), Kristallinität, Langperiode, Kristallicke, Zusammenhang mit Temperatur, Vergleich mit Theorie

\chapter{Fazit}
TODO: Fazit, mögliche Verbesserungen (Herstellung der Probe?)

%ENDE INHALT
\cleardoublepage{}
% Eintrag fürs Inhaltsverzeichnis
\newpage
\begin{thebibliography}{100}
  \bibitem{GefahrenLaser} \url{http://de.wikipedia.org/w/index.php?title=Laser&oldid=128632514#Gefahren}
\end{thebibliography}

\cleardoublepage{}
% Eintrag fürs Inhaltsverzeichnis
% Abbildungsverzeichnis einfügen
\end{document}
