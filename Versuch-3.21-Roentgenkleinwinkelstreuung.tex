% Klassifiziert den Dokumenten-Typ
% Doku: http://exp1.fkp.physik.tu-darmstadt.de/tuddesign/
% Farben: http://www.tu-darmstadt.de/media/medien_stabsstelle_km/services/medien_cd/das_bild_der_tu_darmstadt.pdf
%  bigchapter: Chapter haben doppelte Schriftgröße
%  linedtoc: Linien im Inhaltsverzeichnis wie bei Überschriften
%  colorbacktitle: Der Dokumenten-Titel wird mir der Accentfarbe hinterlegt
\documentclass[bigchapter,colorback,accentcolor=tud4b,linedtoc,11pt]{tudreport}

% Input Dokument hat das Encoding UTF-8
\usepackage[utf8]{inputenc}
% Wichtiges Paket für Links und verlinktes Inhaltsverzeichnis
\usepackage[ngerman]{hyperref}
% Paket für Fußnoten
\usepackage[stable]{footmisc}
% Paket für amsmath (aligned mathe formeln)
\usepackage{amsmath}
% Paket für Bibliotheks-Verzeichnis, square: Verwende eckige statt runde klammern
% \usepackage[square]{natbib}
% Paket zum Plotten von Datensätzen
\usepackage{pgfplots}
\pgfkeys{
  /pgfplots/PolarStyle/.style={
    ylabel=Leistung in W,
    width=0.33\linewidth,
    height=0.33\linewidth,
    scale only axis,
    grid=both,
    tick align=outside,
    tickpos=left,
    minor x tick num=2,
    minor y tick num=1
  }
}

% Anhänge für Original-Messdaten
\usepackage{fancyvrb}

% redefine \VerbatimInput
\RecustomVerbatimCommand{\VerbatimInput}{VerbatimInput}%
{fontsize=\footnotesize,
 %
 frame=lines,  % top and bottom rule only
 framesep=2em, % separation between frame and text
 fontsize=\scriptsize,
 %
 labelposition=topline,
 %
 commandchars=\|\(\), % escape character and argument delimiters for
                      % commands within the verbatim
 commentchar=*        % comment character
}

% Polar Plots
\usetikzlibrary{pgfplots.polar}
% Verwende deutsche Bezeichner für Inhaltsverzeichnis, ... (ngerman = New German: neue Rechtschreibung)
\usepackage{ngerman}
% Deutsche Zahlen (entfernt z.B. das Leerzeichen nach einem Dezimal-Komma)
\usepackage{ziffer} 

\usepackage[verbose]{placeins}

%wegen Grafikverschiebung hinzugefügt
\usepackage{float}

%\usepackage{graphicx}
%\usepackage{caption}
\usepackage{subcaption} %Für subfigures

% PDF-Optionen
\hypersetup{
  pdftitle={TU Darmstadt \- Physikalisches Praktikum für Fortgeschrittene},
  pdfauthor={Esra Bauer und Sören Link},
  pdfsubject={Versuch 3.21},
  pdfview=FitH,
}
% Nummeriere formeln in Subsections einzeln
% Kleines makro zur assymetrischen Fehlerangabe

% Entspricht-Zeichen
\usepackage{scalerel}

\newcommand\equalhat{%
\let\savearraystretch\arraystretch
\renewcommand\arraystretch{0.3}
\begin{array}{c}
\stretchto{
    \scalerel*[\widthof{=}]{\wedge}
    {\rule{1ex}{3ex}}%
}{0.5ex}\\ 
=%
\end{array}
\let\arraystretch\savearraystretch
}
%BEGINN TITELSEITE

\title{Röntgenkleinwinkelstreuung an teilkristallinen Polymeren}

\subtitle{Esra Bauer  \\Sören Link}

\subsubtitle{Betreuer: Jan Gabriel \hfill Versuchsdatum: 24. November 2014}

\author{Esra Bauer, Sören Link}

%\settitlepicture{img/title.jpg}

\institution{Physikalisches Praktikum \\für Fortgeschrittene \\ Versuch 3.21}

\date{\today}


%ENDE TITELSEITE

\begin{document}
%ANFANG DOKUMENT

%Titelseite einfügen
\maketitle

%Inhaltsverzeichnis einfügen
\tableofcontents

%ANFANG INHALT

\chapter{Einleitung}
TODO: Einleitung zur Röntgenkleinwinkelstreuung

\chapter{Grundlagen}
\section{Polymere}
TODO: Grundlagen Polymere (langkettig etc.)

\section{Glasübergang}
TODO: Glasübergang vs. Kristallisation

\section{Kristallinität in Polymeren}
TODO: Warum kristallisieren polymere? Keimbildung, Teilkristallinität, $F = U - TS$ (Van der Vaals-Kräfte), Sphärolite, warum keine vollständige Kristallisation?

Thermoplasten, zu denen auch das PET gehört, welches untersucht werden soll, liegen in der Schmelze als Knäuelstruktur vor, die Molekülketten sind also ungeordnet. Beim Abkühlen ordnen sich diese teilweise an, stark abhängig von der Abkühlgeschwindigkeit und -temperatur. Man spricht dann von Kristallisation. Energetisch am günstigen wäre ein Zustand, in dem sämtliche Ketten vollständig entfaltet parallel aneinanderliegen würden. Dies lässt sich dadurch erklären, dass zwischen den Ketten Van-der-Waals-Kräfte wirken, die vor allem bei kleinem Teilchenabstand zum Tragen kommen und bei größerem Abstand vernachlässigbar schwach werden. Folglich erfährt eine Kette, die nicht parallel zur nächsten liegt, in der Regel Kräfte verschiedener anderer Ketten in verschiedene Richtungen, d.h. der Zustand ist nicht stabil. Liegen die Ketten allerdings parallel, wird die Van-der-Waals-Kraft pro Kette auf ihren Nachbarn maximal, da jedes Teilchen den kleinstmöglichen Abstand zum entsprechenden Teilchen der Nachbarkette hat.

Zunächst erfolgt eine Keimbildung, d.h. es bilden sich sog. Kristallite (lokale Ordnungszustände mit einer typischen Größe von 15-100 nm), in denen die Molekülketten großteils parallel liegen. Großteils deshalb, weil sich die langen Molekülketten, obwohl dies energetisch am günstigsten wäre, nicht vollständig entfalten können. In der Realität falten sich die Ketten und liegen an den Enden und Schlaufen oft ungeordnet vor. Diese Kristallite können nun weiter wachsen, wobei zu beachten ist, dass sie dies bevorzugt in eine ausgezeichnete Richtung tun, da zum einen an den Seiten, an denen die Schlaufen liegen, kein Wachstum stattfinden kann, da hier amporhe Bereiche außenliegend sind, und zum andern die Richtung des größten Temperaturgradienten bevorzugt wird. Somit wächst der Kristallit in die Länge, es entstehen sog. Lamellen, also längliche Kristallstrukturen, in denen die Molekülketten quer zur Längsachse der Lamellen angeordnet sind.

Beim weiteren Wachstum muss unterschieden werden, unter welchem Temperaturbedingungen die Kristallisation erfolgt. Besteht ein starkes Temperaturgefälle, ordnen sich die Lamellen parallel an. Dies ist aber im Rahmen des Versuchs nicht weiter von Belang, da die Proben unter weitgehend isotropen Temperaturbedingen kristallisiert wurden, wobei sich die Lamellen radialsymmetrisch anordnen. Man spricht bei diesen radialsymmetrischen Strukturen von Sphärolithen. Eine vollständige Kristallisation ist in Polymeren nicht möglich, da bereits die Grundbausteine, die Kristallite, amorphe Randbereiche besitzen, an denen sich keine geordnete Struktur anlagern kann. Die Kristallisation lässt sich aber durch Additive signifikant steigern, auch Verunreinigungen und nicht ganz aufgeschmolzene kristalline Bestandteile können die Kristallisation verbessern.

\section{Röntgenstreuung}
TODO: Funktionsweie Röntgenröhren, Peaks, Streuung (Amplitude, Intensität, Furier, Autokorrelationsunfktion, Elektronendichte)

Unter Röntgenstrahlung versteht man elektromagnetische Strahlung hoher Energie (zwischen 100 eV und einigen MeV), deren Wellenlängen entsprechend zwischen 10$^{-12}$ und 10$^{-8}$ liegen. Sie lässt sich beispielsweise mit einem Synchrotron oder einer Röntgenröhre erzeugen, wobei im Folgenden letztere verwendet wird und daher hier kurz auf die Funktionsweise eingegangen werden soll. Im Wesentlichen besteht die Röntgenröhre aus einer beheizten Kathode, die freie Elektronen erzeugt, sowie aus einer Anode, die gegenüber der Kathode eine Hochspannung von 40kV besitzt. Die Elektronen werden also zur Anode beschleunigt, was in zwei Arten von Strahlung resultiert, da zum einen Wechselwirkung der Elektronen mit dem elektromagnetischen Feld der Anode stattfindet, wobei die Elektronen abgelenkt bzw. gebremst werden (dies verursacht ein kontinuierliches Spektrum der sog. Bremsstrahlung), und zum andern die energiereicheren Elektronen Hüllenelektronen aus dem Anodematerial herausschlagen können. Der Effekt ist, dass ein Hüllenelektron der nächsten Schale nachrückt und genau die Energiedifferenz der beiden Schalen in Form von elektromagnetischer Strahlung frei wird. Es entsteht zusätzlich ein diskretes Spektrum der charakteristischen Röntgenstrahlung. Ein Übergang der L-Schale zur K-Schale wird dabei als $K_{\alpha}$-Linie bezeichnet, ein Übergang der M-Schale zur K-Schale als $K_{\beta}$-Linie usw. Die Lage der Linien sind dabei vom Anodenmaterial abhängig, welches in diesem Fall Kupfer ist. 

Alle Messungen werden mit einer sog. Kratky-Kompakt-Kamera durchgeführt. Sie besteht aus einem evakuierbaren Gehäuse (Betriebsdruck ca. 38 mbar), welches einseitig auf der Röntgenröhre aufliegt und an der anderen Seite einen per Schrittmotor fein verschiebbaren Szintillationsdetektor vorgeschaltet hat. An beiden Enden ist das Gehäuse mit 0,25 mm dicken Berylliumfenstern verschlossen, die der Röntgenstrahl durchdringt. Vor der Strahlungsquelle befindet sich zunächst das Kollimationssystem, ein sog. Blockkollimationssystem, welches aus drei rechteckigen, genau geschliffenen Blöcken besteht, wobei zwei Blöcke den Strahl nach oben begrenzen und einer nach unten. Der Strahl passiert zuerst den Eingangspalt (dies ist der vertikale Abstand der in Strahlrichtung ersten beiden Blöcke, welcher auf etwa 80 µm eingestellt wird) und danach den letzten oberen Block, welcher gemeinsam mit dem mittleren Block die Strahlebenen bestimmt und Streustrahlung unterdrückt. So kollimiert, trifft der Strahl auf die Probe (bzw. passiert bei der ersten Messung den leeren Probenhalter) und wird dann auf das Austrittsfenster gestreut. Bei den Messungen mit Probe wird der Primärstrahl, d.h. der nicht gestreute Strahl, durch einen metallenen Beamstop blockiert, um ein direktes Auftreffen auf den Detektor zu verhindern, da dieser dadurch zerstört werden würde. Der Abstand der Probe zum Detektor beträgt 20 cm.

\section{Streuung an zweiphasigen Schichtsystemen}
TODO: Autokorrelationsfunktion, Invariante Q, Herleitung (wie in vorbesprechung), Abweichung Tatsächliche Korrelationsfunktion von einer für ein idealisiertes System


\chapter{Durchführung}
\section{Präparation der PET-Probe}

Bei dem Material, welches in diesem Versuch untersucht wird, handelt es sich um Polyethylenterephtalat (PET), welches man in Form von Getränkeflaschen kennt und auch für die Herstellung von Folien und Fasern verwendet wird. Wir bedienen uns einer handelsüblichen PET-Flasche und schneiden kleine Stücke passend für den Probenhalter aus. Der Probenhalter ist ein längs geteiler Messingzylinder, in dessen Mitte ein längliches Messingplättchen einer Stärke von 1 mm mit rechteckiger Aussparung verschraubt wird. In dieser Aussparung fixieren und schmelzen/kristallisieren wir unsere Probe. Zur Fixierung dient Aluminiumfolie, die einmal um das Plättchen herumgelegt wird, nachdem das PET-Rohmaterial eingelegt worden ist. Nun legen wir das Plättchen auf eine Kochplatte und lassen es einige Zeit bei knapp unter 300 $^{\circ}$C schmelzen. Anschließend kommt die Probe für 10 Minuten (per Stoppuhr genau überprüft) bei 170 $^{\circ}$C in einen Ofen, um sie kristallisieren zu lassen. Nach Ablauf der 10 Minuten wird sie in kaltem Wasser abgeschreckt und ist nun nach Abtrocknung zur Messung bereit. 

Bei der Schmelze ist zu beachten, dass das PET verläuft und man deswegen die Menge nicht zu reichlich bemessen sollte, da es sonst aus der Aussparung überläuft, was in der Tat auch beinahe geschehen ist. Zudem hat sich gegen Ende der Schmelze eine bräunliche Verfärbung eingestellt, was zwei Gründe haben kann; zum einen hat der Schmelzvorgang deutlich länger angedauert als im Anleitungsblatt empfohlen, zum andern ist die Temperatur teilweise knapp über 300 $^{\circ}$C angestiegen. Zur Schmelzdauer ist zu sagen, dass die im Anleitungsblatt empfohlenen 5 Minuten etwas knapp bemessen sind, da die Wärmeleitung zwischen der Heizplatte und dem Probenhalter dazu nicht ausreicht, allerdings hat die tatsächliche Schmelze etwa 45 Minuten angedauert, was vermutlich wiederum zu lang ist (ideal sollten ca. 15-20 Minuten sein). Um die Temperatur konstant zu halten, ist es außerdem erforderlich, die Heizleistung der Platte nachzuregeln, u.a. wohl deswegen, weil zur Fixierung des Temperatursensors ein Ziegelstein auf der Platte liegt, der sich erst nach und nach aufheizt und zu Anfang deshalb mehr Heizleistung aufnimmt. Selbiges gilt auch für die übrigen Materialien, die mit aufgeheizt werden (metallene Heizplatte, Probenhalter usw.).

\section{Bestimmung des Primärstrahlprofils}

Vor Beginn der eigentlichem Messung ist es notwendig, die Intensitätsverteilung des Primärstrahls zu bestimmen, da die Strichkollimation zwar gegenüber der Punktkollimation den Vorteil der generell höheren Intensität besitzt, jedoch das Strahlprofil nicht genau rechteckig ist, sondern abgerundet und zu den Rändern allmählich ausläuft. Aus diesem Grund müssen wir also zuerst eine Messung ohne Probe und ohne Primärstrahlfänger durchführen. Da ohne weitere Maßnahmen damit der empfindliche Szintillitationsdetektor zerstört werden würde, schalten wir den Messingabsorber vor die Strahlungsquelle dazu, der Strahl wird insgesamt also stark abgeschwächt. Zusätzlich montieren wir einen horizontalen Spalt mit einer Breite von 200 µm vor den Detektor. Das Innere der Kratky-Kamera wird anschließend evakuiert und eine winkelabhängige Messung der Intensität gestartet. 

Zuletzt setzen wir den Nullpunkt des Detektors auf das eben bestimmte Maximum des Primärstrahls, um die folgenden Messungen relativ dazu betrachten zu können.

\section{Hintergrundmessung mit leerem Probenhalter und Aluminiumfolie}
Um aus späteren Messungen die vom Probenbehlter und der Aluminiumfolie verursachte Hintergrundstrahlung herausrechnen zu können, war es notwendig, eine Messreihe mit einem leeren Probenbehälter durchzuführen. Dazu wurde der leere, mit Aluminiumfolie umwickelter, Probenbehälter in der Apparatur installiert, die Kratky-Kamera evakuiert und erneut eine winkelabhängige Messung der Intensität gestartet. Im Gegensatz zur Bestimmung des Primärstrahlprofils messen wir hier allerdings nur die gestreute Röntgenstrahlung, es wird also kein Absorber zur Abschwächung des Primärstrahls verwendet. Um dabei das Szintillatormessgerät nicht zu beschädigen, wird ein Primärstrahlfänger angebracht. Dieser ist so angebracht, dass der nicht abgelenkte Primärstrahl vollständig geblockt wird, die gestreute Strahlung allerdings ungehindert in den Szintillator eindringen kann.

Die Original-Messdaten für diese Messung sind als "`Untergrund.0.txt"' angehängt.

\subsection{Wanderspaltmessung}
Zur Entschmierung der einzelnen Streudatenmessungen ist es notwendig, noch eine sogenannte Wanderspaltmessung durchzuführen. Dabei wird der zuvor angebrachte Horizontaler $200\mu m$ Spalt durch einen ebenfalls unbeweglichen $32 \mu m$ breiten vertikalen Spalt ausgetauscht. Zusätzlich wird vor der Probe ein beweglicher Spalt eingebracht. Dieser fährt in einem Zeitraum von 10 Sekunden von einer Seite des Primärstrahls zur anderen und anschließend mit gleicher Geschwindigkeit wieder zurück. Da bei dieser Messmethode nur ein sehr kleiner Teil der Strahlung beide Spalte passieren kann und statt der gesreuten Strahlung die Intensität des Primärstrahls gemessen wird, muss für die Wanderspaltmessung der Primärstrahlfänger entfernt werden.

Die Messung der Intensität erfolgt hierbei nicht winkelabhängig, statdessen wird pro Messung die Intensität des Primärstrahls über die gesamten 20 Sekunden aufgenommen den der bewegliche Spalt braucht, um wieder in die Startposition zurückzukehren.

Um den Fehler der Messung (Verursacht durch die statistische Natur der emittierten Röntgenstrahlung oder nicht gleichzeitigen Startens der Messung und des Spaltmotors) zu reduzieren, wird für jede Probe die Wanderspaltmessung 5 mal durchgeführt.

Die Original-Messdaten für diese Messung sind als "`Alu.ms"' angehängt.

\section{Messungen mit Probe}
Analog zur winkelabhängigen- und Wanderspaltmessung mit der leeren Probe haben wir die Intensität des Primärstrahls mit der Wanderspalt-Messmethode sowie die winkelabhängige Intensität der gestreuten Strahlung für jede der uns zur Verfügung stehendenden Proben gemessen. Dabei ist anzumerken, dass bei der Messung der ersten Probe die Röntgenröhre ausgeschaltet war. Die Ursache für die Abschaltung ist uns nicht genau bekannt, eventuell ist jemand trotz aller Vorsicht versehentlich gegen den gelben Strom-Schalter für die Röntgenröhre gekommen.

Weiterhin wurde bei der Wanderspaltmessung wiederholt die Verbindung zwischen Messgerät und dem zur Auswertung benutzten Computer getrennt. Dies wurde vermutlich durch eine kleine Spannungsspitze beim Ein- oder Ausschalten des Motors für den beweglichen Spalt verursacht.

Die einzelnen Messungen wurde von uns in folgender Reihenfolge durchgeführt:


\begin{center}
  \begin{tabular}{|p{2.2cm}|p{4.5cm}|p{3cm}|p{4cm}|}
    \hline
    Messungs-Nummer & Kristallisationstemperatur der Probe in °C & Messmethode    & Name der Messdaten-Datei \\ \hline
    1               & 140                                        & Winkelabhängig & PET-140.0.txt \\ \hline
    2               & 140                                        & Wanderspalt    & PET-140.ms \\ \hline
    3               & 170                                        & Wanderspalt    & PET-170.ms \\ \hline
    4               & 170                                        & Winkelabhängig & PET-170.0.txt \\ \hline
    5               & 170 (Selbst getempert)                     & Winkelabhängig & PET-170-eigen.0.txt \\ \hline
    6               & 170 (Selbst getempert)                     & Wanderspalt    & PET-170-eigen.ms \\ \hline
    7               & 200                                        & Wanderspalt    & PET-200.ms \\ \hline
    8               & 200                                        & Winkelabhängig & PET-200.0.txt \\ \hline
    9               & 215                                        & Winkelabhängig & PET-215.0.txt \\ \hline
    10              & 215                                        & Wanderspalt    & PET-215.ms \\ \hline
	\end{tabular}
\end{center}

\chapter{Auswertung}
\section{Strahlcharakterisierung}
TODO: Darstellung des Primärstrahls

\section{PET-Streudaten}
TODO: Rohdaten, Entschmierung, Intensität vs. Streuvektor, Langperiode abschätzen

\section{Eindimensionale Korrelationsfunkton}
TODO: Berechnung/Darstellung von K(Z), Kristallinität, Langperiode, Kristallicke, Zusammenhang mit Temperatur, Vergleich mit Theorie

\chapter{Fazit}
TODO: Fazit, mögliche Verbesserungen (Herstellung der Probe?)

%ENDE INHALT
\cleardoublepage{}
% Eintrag fürs Inhaltsverzeichnis
\newpage
\begin{thebibliography}{100}
  \bibitem{GefahrenLaser} \url{http://de.wikipedia.org/w/index.php?title=Laser&oldid=128632514#Gefahren}
\end{thebibliography}

\chapter{Messdaten}
\VerbatimInput[label=\fbox{beamprofile.bp}]{data/beamprofile.bp}
\VerbatimInput[label=\fbox{untergrund.0.txt}]{data/untergrund.0.txt}
\VerbatimInput[label=\fbox{Alu.ms}]{data/Alu.ms}
\VerbatimInput[label=\fbox{PET-140.0.txt}]{data/PET-140.0.txt}
\VerbatimInput[label=\fbox{PET-140.ms}]{data/PET-140.ms}
\VerbatimInput[label=\fbox{PET-170.0.txt}]{data/PET-170.0.txt}
\VerbatimInput[label=\fbox{PET-170.ms}]{data/PET-170.ms}
\VerbatimInput[label=\fbox{PET-170-eigen.0.txt}]{data/PET-170-eigen.0.txt}
\VerbatimInput[label=\fbox{PET-170-eigen.ms}]{data/PET-170-eigen.ms}
\VerbatimInput[label=\fbox{PET-200.0.txt}]{data/PET-200.0.txt}
\VerbatimInput[label=\fbox{PET-200.ms}]{data/PET-200.ms}
\VerbatimInput[label=\fbox{PET-215.0.txt}]{data/PET-215.0.txt}
\VerbatimInput[label=\fbox{PET-215.ms}]{data/PET-215.ms}
\cleardoublepage{}
% Eintrag fürs Inhaltsverzeichnis
% Abbildungsverzeichnis einfügen
\end{document}
